% Chapter 2: Principles of Phase-Locked Loops
\chapter{Mathematical Fundamentals of Phase-Locked Loops}

Here comes some introducing text.

% Section 2.1: 
\section{Phase-Locked Loops}

% Subsection 2.1.1: 
\subsection{General Purpose}

- Wozu braucht man Phase Locked Loops?

\noindent - Wie werden sie technisch realisiert? -> Quellen angeben!

\noindent - Zeige den Regelkreis auf -> Danach: Link zu Kalman Filtering.

% Subsection 2.1.2: 
\subsection{The Kalman View on Phase Locked Loops}

- Wie funktioniert ein Kalman Filter im Allgemeinen? -> Und wieder: Quellen!

\noindent - Wie sieht ein PLL mit Kalman Filter Implementierung aus? -> Block-Schaltbild

\noindent - Verweise auf nächste Section und kündige an, dass nun die Theorie entwickelt wird, um einen PLL zu implementieren


% Section 2.2: 
\section{The Factor Graph Approach}

% Subsection 2.2.1: 
\subsection{Discrete State Space Model for Harmonic Signals}

So far, we have investigated PLLs only conceptually without giving any insights into concrete mathematical details. We now do so by starting with the notion of \emph{harmonic signals}. The general representation of a harmonic signal $s(t)$ is the following \cite{somethingToCite}

\begin{equation}
\label{eq: harmonic signal raw}
s(t) = \sum_{k=0}^N a_k \cos \left( 2 \pi f_k t + \varphi_k \right) .
\end{equation}

The lowest frequency $f_0$ is usually called the \emph{fundamental frequency} and the higher frequencies $f_k,\; k=1, \ldots, N$ are the \emph{upper harmonics} and are integer multiples of $f_0$, i.e.,

\begin{equation*}
\label{eq: upper harmonics}
f_k = (k+1) f_0, \qquad k = 1, \ldots, N.
\end{equation*}





\noindent - Beginne mit continous time model der harmonischen Signale.

\noindent - Verweise auf Nyquist Theorem und stelle fest, dass harmonische Signale ganz toll abgetastet werden können, wenn $f_s \geq 2f_g$, wobei $f_s$ die Abtastfrequenz ist und $f_g$ die höchste Frequenz im harmonischen Signal -> Noch was? Genau: QUELLEN!

\noindent - Konstruiere anschliessend ein diskretes Model.

\noindent - Erstelle nun das state space model.

\noindent - Füge noise hinzu -> Danach: Link zum minimization problem


% Subsection 2.2.2: 
\subsection{The Phase-Locked Loop as a Minimization Problem}

- Formuliere nun den PLL als minimization problem, d.h. erstelle die cost function

\noindent - Erweitere die cost function um den decay factor

\noindent - Forme das Ganze um, bis die cost function faktorisiert ist -> Danach: Link zu Solution via Gaussian Message Passing

% Subsection 2.2.3: 
\subsection{Solution via Gaussian Message Passing}

- Zeichne den factor graph zu der betrachteten cost function -> Q-U-E-L-L-E-N!!!

\noindent - Zeige auf, dass belief propagation, d.h. Max-product message passing das problem löst.

\noindent - Führe Berechnungen durch und rechne $\messF{W}{X}{}$ und $\messF{W}{X}{}\messF{m}{X}{}$

\noindent - Füge den Pseudocode hinzu

\noindent - Erwähne kurz, dass man aus factor graphs interessante Eigenschaften herauslesen kann -> Danach: Link zu Properties of Gaussian Message Passing

% Subsection 2.2.3:  
\subsection{Properties of Gaussian Message Passing}

- Füge explizite Formeln für die messages hinzu

\noindent - Zeige, dass $\messF{m}{X}{}$ unabhängig ist vom noise

\noindent - Berechne $\messF{V}{X}{}$ und $\messF{m}{X}{}$ und zeige ultimativ auf, dass Kalman filtering und factor graph ein und dasselbe sind -> Wie geil ist denn das?

\noindent - Berechne steady state precision matrix und erwähne, dass dies nun computation power spart -> Herleitung oder die berühmt berüchtigen QUELLEN!







\clearpage